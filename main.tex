\documentclass{beamer}
\usetheme{Rochester}

\title[F3ildCrypt]{F3ildCrypt: End-to-End Protection of Sensitive Information
in Web Services}

\author[Burnside, Keromytis]{Matthew Burnside and Angelos D. Keromytis}

\institute[Columbia University]{
Department of Computer Science\\
Columbia University\\
\texttt{\{mb, angelos\}@cs.columbia.edu}
}
\date{ISC 2009}

\begin{document}

\begin{frame}[plain]
    \titlepage
\end{frame}

\begin{frame}
\frametitle{Motivation}
\begin{itemize}
\item Identity-related information is valuable
\item You must provide such information when using an online merchant
\item This information is vulnerable to disclosure at many points
\item Can we protect this information end-to-end?
\end{itemize}
\end{frame}

\begin{frame}
\frametitle{Outline}
\tableofcontents
\end{frame}

\section{Introduction}

\begin{frame}
\frametitle{Introduction}
\begin{itemize}
\item Users have to trust online merchants:
\item Merchant is not malicious
\item Merchant site is maintained by diligent sysadmins
\item Merchant will protect always sensitive information
\end{itemize}
\end{frame}

\begin{frame}
\frametitle{SOA trust}
\begin{itemize}
\item In Service Oriented Architectures, have to trust:
\item Merchant \emph{and peer SOAs} are not malicious
\item Merchant \emph{and peer SOAs} are maintained by diligent sysadmins
\item Merchant \emph{and peer SOAs} will always protect sensitive information
\end{itemize}
\end{frame}

\begin{frame}
\frametitle{Data in transit}
\begin{itemize}
\item In this work, we focus on data in transit
\item We protect the data from the web browser to the back-end database
\end{itemize}
\end{frame}

\begin{frame}
\frametitle{Example}
\begin{itemize}
\item XXX: Diagram showing web browser, merchant, and SOA doing credit-card
transactions.  Even with SSL, only protected from web browser to merchant.
\end{itemize}
\end{frame}

\section{Related work}

% \begin{frame}
% \frametitle{XACML}
% \begin{itemize}
% \item eXtensible Access Control Markup Language
% \item XML standard for defining policies, requests, and reponses
% \end{itemize}
% \end{frame}

\begin{frame}
\frametitle{Proxy re-encryption}
For all plaintext $P$, Alice: $(pk_A, sk_A)$, Bob: $(pk_B, sk_B)$:
\begin{equation*}
pk_B(p) = rk_{A \to B}( pk_A (P))
\end{equation*}
\begin{itemize}
\item \cite{atomic_proxy_reencryption} 
\item \cite{proxy_reencryption}
\end{itemize}
\end{frame}

\begin{frame}
\frametitle{W3bCrypt}
\begin{itemize}
\item Introduced end-to-end encryption in web pipelines
\item Firefox plugin for application-level crypto
\item Requires disclosure of corporate network details
\end{itemize}
\end{frame}

\section{Architecture}
\begin{frame}
\frametitle{Architecture}
\begin{itemize}
\item Design goals
\item F3ieldCrypt architecture
\item Example session
\end{itemize}
\end{frame}

\begin{frame}
\frametitle{Network model}
\begin{itemize}
\item SOA-style network
\item Each SOA may have multiple child SOAs
\item SOAs wish to prevent disclosure of logical architecture and peering 
\end{itemize}
\end{frame}

\begin{frame}
\frametitle{Threat model}
\begin{itemize}
\item XXX
\end{itemize}
\end{frame}

\begin{frame}
\frametitle{Design goals}
\begin{itemize}
\item End-to-end protection of XML fields -- even across SOA boundaries
\item Confidentiality of logical architecture of each SOA must be respected
\end{itemize}
\end{frame}

\begin{frame}
\frametitle{F3ieldCrypt architecture}
\begin{itemize}
\item Each SOA $s$ publishes a public key $pk_{E_s}$
\item Browser $b$ generates plaintext $P$
\item $b$ sends $C = pk_{E_s}(P)$ 
\item At $s$, proxy re-encrypt $C$ to internal hosts and child SOAs $0...n$ 
\end{itemize}
\end{frame}

\begin{frame}
\frametitle{Key generation}
\begin{itemize}
\item Key pair $pk_{E_s}, sk_{E_s}$ generated at the \alert{external-key holder}
\item Public keys of applications $pk_{I_0}...pk_{I_n}$ are collected
\item Used in conjunction with $sk_{E_s}$ to generate $rk_{E \to I_0}...rk_{E
\to I_n}$
\end{itemize}
\end{frame}

\begin{frame}
\frametitle{Example session}
\begin{itemize}
\item
\end{itemize}
\end{frame}

\begin{frame}
\frametitle{Evaluation}
\begin{itemize}
\item
\end{itemize}
\end{frame}

\section{Evaluation}
\begin{frame}
\frametitle{Evaluation}
\begin{itemize}
\item
\end{itemize}
\end{frame}

\section{Evaluation}
\begin{frame}
\frametitle{Evaluation}
\begin{itemize}
\item
\end{itemize}
\end{frame}

\section{Conclusion}
\begin{frame}
\frametitle{Conclusion}
\begin{itemize}
\item
\end{itemize}
\end{frame}

\begin{thebibliography}{10}
\frametitle{References}
\bibitem[Ateniese et al., 2005]{proxy_reencryption}
G.~Ateniese, K.~Fu, M.~Green, and S.~Hohenberger.
\newblock Improved proxy re-encryption schemes with applications to secure
  distributed storage.
\newblock In {\em Proceedings of the 12th Annual Network and Distributed
  Systems Security Symposium (NDSS 2005)}, 2005.

\bibitem[Blaze et al., 1998]{atomic_proxy_reencryption}
Matt Blaze, G.~Bleumer, and M.~Strauss.
\newblock Divertible protocols and atomic proxy cryptography.
\newblock In {\em Proceedings of Eurocrypt '98}, pages 127--144, 1998.

\bibitem[Stavrou et al., 2006]{w3bcrypt}
Angelos Stavrou, Michael Locasto, and Angelos Keromytis.
\newblock W3bcrypt: Encryption as a stylesheet.
\newblock In {\em Proceedings of the 4th Applied Cryptography and Network
  Security Conference (ACNS 2006)}, pages 349--364, 2006.

\end{thebibliography}

\end{document} 
